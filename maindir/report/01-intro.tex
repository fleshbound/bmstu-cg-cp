\chapter*{ВВЕДЕНИЕ}
\addcontentsline{toc}{chapter}{ВВЕДЕНИЕ}

Компьютерная графика в 21 веке не перестала быть развивающейся наукой и упоминается в \cite{залогова2005компьютерная,порев2002компьютерная,тозик2013инженерная,митин2016компьютерная,турлюн2014компьютерная}. Одной из стандартных задач компьютерной графики является создание изображения с использованием методов удаления невидимых линий, закраски \cite{порев2002компьютерная}.

За последние 20 лет визуализация зеркальных поверхностей (зеркал) остается актуальной проблемой, о чем свидетельствует ее обсуждение в \cite{lensch2005realistic,reshetouski2013mirrors,miguel2014real,hiranyachattada2021demonstration}.

Цель работы --- разработка программного обеспечения для визуализации изображения объекта в неотполированном цветном зеркале.

Для достижения поставленной цели требуется решить следующие задачи:

\begin{enumerate}
	\item проанализировать предметную область зеркальных поверхностей и существующие методы их визуализации;
	\item описать выбранные в ходе анализа алгоритмы визуализации;
	\item реализовать описанные алгоритмы визуализации;
	\item исследовать реализованные алгоритмы.
\end{enumerate}