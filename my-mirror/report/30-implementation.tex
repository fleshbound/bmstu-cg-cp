\chapter{Технологическая часть}

В данном разделе описывается и обосновывается набор средств реализации программного обеспечения и приводятся детали реализуемой программы.

\section{Средства реализации программного обеспечения}

В качестве языка программирования был выбран язык C++, так как присутствуют следующие причины:

\begin{itemize}
	\item имеется опыт разработки на данном языке программирования;
	\item средствами языка можно реализовать все спроектированные алгоритмы.
\end{itemize}

Дополнительно используются приведенные ниже библиотеки:

\begin{itemize}
	\item кроссплатформенная библиотека Qt, предназначенная для разработки графического интерфейса (выбранная по причине наличия опыта работы с использованием данной библиотеки) \cite{qt};
	\item библиотека Intel® Threading Building Blocks (Intel® TBB), предназначенная для создания параллелизма на уровне инструкций во время выполнения программы для эффективного использования ресурсов процессора \cite{inteltbb} (выбранная по причине наличия необходимости реализации параллельной версии алгоритма обратной трассировки лучей и совместимости с другими библиотеками потоков \cite{inteltbb}, что говорит об отсутствии конфликтов с потоками Qt);
	\item библиотека stl\_reader, предназначенная для обработки файлов, хранящих данные в формате STL, и преобразования их содержимого в пользовательские контейнеры \cite{stlreader} (выбранная по причине необходимости чтения STL-файлов заранее подготовленных моделей геометрических тел).
\end{itemize}

Для ускорения сборки программного обеспечения использовалась утилита CMake \cite{cmake}.

\section{Структура программного обеспечения}

В разрабатываемом программном обеспечении реализуются классы, описанные в диаграмме классов, которая представлена на рисунках \ref{img:uml_classes}, \ref{img:uml_classes_1},~\ref{img:uml_classes_2}.

\includeimage{uml_classes}{f}{h!}{0.9\textwidth}{Диаграмма классов (часть 1)}

\includeimage{uml_classes_1}{f}{h!}{0.9\textwidth}{Диаграмма классов (часть 2)}

\includeimage{uml_classes_2}{f}{h!}{0.9\textwidth}{Диаграмма классов (часть 3)}

Основные классы, реализуемые в программном обеспечении:

\begin{enumerate}
	\item {
		классы, служащие для управления сценой пользователем:
		\begin{itemize}
			\item Scene --- класс, описывающий набор объектов сцены;
			\item SceneManager --- класс инкапсулирования сцены;
			\item QtDrawer --- класс, предназначенный для изменения объекта QPixmap (экземпляр класса библиотеки Qt \cite{qt})
			\item Facade --- реализация паттерна <<фасад>>;
			\item NameCommand, SceneCommand --- реализация паттерна <<команда>>.
		\end{itemize}
}
	\item {
		классы, служащие для описания объектов сцены:
		\begin{itemize}
			\item Object --- базовый класс объекта сцены, определяющий его интерфейс;
			\item Model, Triangle --- базовые классы модели объекта сцены и ее декомпозиции;
			\item Camera --- наблюдатель (камера);
			\item Light --- точечный источник света;
		\end{itemize}
}
	\item {
		классы, обеспечивающие реализацию алгоритма обратной трассировки лучей: Ray, KDTree, BBox, Material, QVector.
}
\end{enumerate}

\clearpage

\section{Пользовательский интерфейс}

Интерфейс пользователя включает в себя следующие объекты:

\begin{itemize}
	\item область изображения сцены;
	\item {поля для изменения параметров зеркала:
	\begin{itemize}
		\item списки доступных видов и цветов;
		\item поле <<Коэффициент диффузного отражения>> (вещественное число в диапазоне от 0 до 1);
		\item поле <<Степень полировки>> (целое число от 0 до 100);
		\item поле <<Коэффициент масштабирования радиуса кривизны>> (вещественное число в диапазоне от 0 до 100);
		\item кнопка <<Применить изменения>>;
	\end{itemize}
}
	\item {поля для изменения параметров предмета:
	\begin{itemize}
		\item списки доступных видов и количества граней (при наличии);
		\item поле <<Коэффициент масштабирования высоты предмета>> (вещественное число в диапазоне от 0 до 100);
		\item поле <<Коэффициент масштабирования радиуса основания или окружности, описанной около основания>> (вещественное число в диапазоне от 0 до 100);
		\item поля <<kx>>, <<ky>> для изменения коэффициента угла наклона оси симметрии (вещественные числа в диапазоне от -5 до 5);
		\item кнопка <<Применить изменения>>;
	\end{itemize}
}
	\item {поля для изменения параметров точечного источника света:
	\begin{itemize}
		\item поля <<dx>>, <<dy>>, <<dz>> для смещения положения источника света (вещественные числа в диапазоне от -1000 до 1000);
		\item список доступных цветов;
		\item кнопка <<Применить изменения>>;
	\end{itemize}
}
	\item {поля для изменения параметров камеры:
		\begin{itemize}
			\item поля <<dx>>, <<dy>>, <<dz>> для смещения камеры (вещественные числа в диапазоне от -1000 до 1000);
			\item поля <<ax>>, <<ay>>, <<az>> для поворота камеры (вещественные числа в диапазоне от -1000 до 1000);
			\item кнопка <<Применить изменения>>.
		\end{itemize}
	}
\end{itemize}

На рисунке \ref{img:interface_example} представлен пользовательский интерфейс разработанного программного обеспечения.

\includeimage{interface_example}{f}{h!}{0.9\textwidth}{Пользовательский интерфейс}

\section*{Вывод}

В данном разделе был описан и обоснован набор средств реализации программного обеспечения, приведены детали реализуемой программы и описан пользовательский интерфейс.