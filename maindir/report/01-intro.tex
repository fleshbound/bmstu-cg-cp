\chapter*{ВВЕДЕНИЕ}
\addcontentsline{toc}{chapter}{ВВЕДЕНИЕ}

Компьютерная графика в 21 веке не перестала быть развивающейся наукой и упоминается в \cite{залогова2005компьютерная}, \cite{порев2002компьютерная}, \cite{тозик2013инженерная}, \cite{митин2016компьютерная}, \cite{турлюн2014компьютерная}. 
Одной из стандартных задач компьютерной графики является синтез изображения \cite{порев2002компьютерная}, \cite{куров}.

За последние 20 лет визуализация зеркальных поверхностей (зеркал) остается актуальной проблемой, о чем свидетельствует ее обсуждение в \cite{lensch2005realistic}, \cite{reshetouski2013mirrors}, \cite{miguel2014real}, \cite{hiranyachattada2021demonstration}.

Цель работы --- разработка программного обеспечения для моделирования статической сцены с геометрическим телом и его изображением в зеркале.

Для достижения поставленной цели требуется решить следующие задачи:

\begin{enumerate}
	\item проанализировать предметную область зеркальных поверхностей, рассмотреть известные методы и алгоритмы решения задачи синтеза изображения в контексте моделирования статической сцены;
	\item спроектировать программное обеспечение;
	\item выбрать средства реализации и разработать программное обеспечение;
	\item исследовать характеристики разработанного программного обеспечения.
\end{enumerate}