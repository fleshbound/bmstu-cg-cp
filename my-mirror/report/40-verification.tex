\chapter{Исследовательская часть}

В данном разделе описывается цель исследования, технические характеристики устройства, на котором проводится исследование, приводятся результаты исследования и их анализ.

\section{Цель исследования, технические характеристики устройства}

В реализации алгоритма обратной трассировки лучей учитывается коэффициент диффузного отражения поверхности зеркала.
В том случае, когда он равен единице, рекурсивная часть реализованного алгоритма не выполняется и отражение предмета не формируется.

Цель исследования заключается в определении наличия зависимости времени выполнения алгоритма от коэффициента диффузного отражения.

Технические характеристики устройства, на котором проводится тестирование:
\begin{itemize}
	\item операционная система --- Ubuntu 22.04.3 LTS;
	\item объем оперативной памяти --- 8 Гб;
	\item процессор --- Intel® Core™ i5-8350U CPU @ 1.70GHz × 8.
\end{itemize}

Тестирование проводится на устройстве, подключенном к сети электропитания.
Во время тестирования устройство нагружается только встроенными приложениями окружения, окружением, системой тестирования.

Используется компилятор g++.
Оптимизация отключалась с помощью использования опции O0.

\section{Результаты тестирования}

Замеры времени выполнения для каждого значения коэффициента диффузного отражения выполняются 20 раз.
Максимальное значение коэффициента --- 1, изменение значения выполнялось с шагом 0,1, начиная с 0,1.

В таблице \ref{tab:time} приведены замеры времени выполнения программы в отношении к времени выполнения программы в случае равенства значения коэффициента единице.

\begin{table}[H]
	\centering
	\caption{\label{tab:time}Замеры относительного времени выполнения}
	\scalebox{0.75}
	{
		\begin{tabular}{|r|r|}
			\hline \specialcell{Коэффициент зеркального \\ отражения} & \specialcell{Относительное время \\выполнения}\\\hline
			\num{0.1} & \num{1.1015}  \\\hline
			\num{0.2} & \num{1.0965}  \\\hline
			\num{0.3} & \num{1.1006}  \\\hline
			\num{0.4} & \num{1.1014}  \\\hline
			\num{0.5} & \num{1.0993}  \\\hline
			\num{0.6} & \num{1.0980}  \\\hline
			\num{0.7} & \num{1.0996}  \\\hline
			\num{0.8} & \num{1.1004}  \\\hline
			\num{0.9} & \num{1.0983}  \\\hline
			\num{1.0} & \num{1.0000}  \\\hline
		\end{tabular}
}
\end{table}

На рисунке \ref{img:data} представлены дискретные данные, полученные в результате тестирования.

\includeimage{data}{f}{h}{0.9\textwidth}{Данные, полученные в результате замеров относительного времени выполнения}

Зависимость относительного времени работы программы от коэффициента диффузного отражения аппроксимируется с помощью метода наименьших квадратов.

Результат проведения аппроксимации представлен на рисунке \ref{img:approx}.

\includeimage{approx}{f}{h}{0.9\textwidth}{Аппроксимация данных, полученных в результате замеров относительного времени выполнения}

\section*{Вывод}

В данном разделе описывается цель исследования, технические характеристики устройства, на котором проводится исследование, приводятся результаты исследования и их анализ.

Было проведено исследование зависимости времени выполнения программы от значения коэффициента диффузного отражения поверхности зеркала.

В результате аппроксимации с использованием метода наименьших квадратов можно утверждать, что время выполнения программы уменьшается при коэффициенте диффузного отражения, равном единице.
В остальных случаях время выполнения программы от коэффициента не зависит.