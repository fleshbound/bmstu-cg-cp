\chapter{Конструкторская часть}

В данном разделе формально описаны функциональная декомпозиция разрабатываемого ПО и используемые алгоритмы, даны описания структур данных.

\section{Функциональная декомпозиция}

На рисунке \ref{img:task_diagram_a1} представлена диаграмма IDEF0 результата функциональной декомпозиции разрабатываемого программного обеспечения.

\includeimage{task_diagram_a1}{f}{h}{0.9\textwidth}{Диаграмма IDEF0 результата функциональной декомпозиции ПО}

Камера --- структура данных, содержащая информацию о наблюдателе: местоположение, направления взгляда и вертикальной оси.
Дополнительной информацией являются отношение сторон используемого экрана и область видимости (угол fov), необходимые для обеспечения возможности применения алгоритма обратной трассировки лучей.

Используемая модель камеры представлена на рисунке \ref{img:camera}.



В алгоритме обратной трассировки лучей необходимо, чтобы была известна информация о точке пересечения луча с некоторой поверхностью (для обработки вторичных лучей) и ее спектральных характеристиках (для выполнения вычислений по формуле \ref{eq:intense_whitted}).

Вычислительная стоимость определения пересечений произвольного луча с одним выделенным объектом, как указывают авторы \cite{роджерс}, \cite{боресков}, может оказаться высокой, поэтому для избавления от ненужной части поиска производится проверка пересечения луча с ограничивающим телом объекта.

\textit{Ограничивающим телом} называется некоторое простое геометрическое тело (например, параллелепипед, сфера), описанное около одного или нескольких объектов сцены \cite{боресков}.

Для организации ограничивающих тел объектов сцены используются иерархические структуры, одной из которых является kD-дерево \cite{боресков}.

\textit{kD-дерево} --- бинарное дерево ограничивающих параллелепипедов, вложенных друг в друга \cite{боресков}.

% схема построения дерева

% схема пересечения луча с деревом

% схема алгоритма обратной трассировки

Согласно техническому заданию необходимо изменять положение источника света, то есть дать пользователю возможность выполнить перенос в отношении источника света, который реализовывается с помощью матрицы переноса в трехмерном пространстве, представленной в формуле \ref{eq:matrix_move} \cite{куров}.

\begin{equation}\label{eq:matrix_move}
	M(dx, dy, dz) = \begin{pmatrix}
		1 & 0 & 0 & 0 \\
		0 & 1 & 0 & 0 \\
		0 & 0 & 1 & 0 \\
		dx & dy & dz & 1
	\end{pmatrix}
\end{equation}

Аналогично перемещение камеры осуществляется с помощью матрицы, представленной в формуле \ref{eq:matrix_move}, а поворот в отношении камеры реализовывается c использованием матриц поворота вокруг осей $x$, $y$, $z$ на угол $\alpha$, которые представлены в формулах \ref{eq:matrix_rotate_x}, \ref{eq:matrix_rotate_y}, \ref{eq:matrix_rotate_z} соответственно.

\begin{equation}\label{eq:matrix_rotate_x}
	M_x(\alpha) = \begin{pmatrix}
		1 & 0 & 0 \\
		0 & cos(\alpha) & -sin(\alpha) \\
		0 & sin(\alpha) & cos(\alpha)
	\end{pmatrix}
\end{equation}

\begin{equation}\label{eq:matrix_rotate_y}
	M_y(\alpha) = \begin{pmatrix}
		cos(\alpha) & 0 & sin(\alpha) \\
		0 & 1 & 0 \\
		-sin(\alpha) & 0 & cos(\alpha)
	\end{pmatrix}
\end{equation}

\begin{equation}\label{eq:matrix_rotate_z}
	M_z(\alpha) = \begin{pmatrix}
		cos(\alpha) & -sin(\alpha) & 0 \\
		sin(\alpha) & cos(\alpha) & 0 \\
		0 & 0 & 1
	\end{pmatrix}
\end{equation}