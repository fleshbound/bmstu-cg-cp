\chapter{Технологическая часть}

В данном разделе описывается и обосновывается набор средств реализации программного обеспечения и приводятся детали реализуемой программы.

\section{Средства реализации программного обеспечения}

В качестве языка программирования был выбран язык C++, так как присутствуют следующие причины:

\begin{itemize}
	\item имеется опыт разработки на данном языке программирования;
	\item средствами языка можно реализовать все спроектированные алгоритмы.
\end{itemize}

Дополнительно используются приведенные ниже библиотеки:

\begin{itemize}
	\item кроссплатформенная библиотека Qt, предназначенная для разработки графического интерфейса (выбранная по причине наличия опыта работы с использованием данной библиотеки) \cite{qt};
	\item библиотека Intel® Threading Building Blocks (Intel® TBB), предназначенная для создания параллелизма на уровне инструкций во время выполнения программы для эффективного использования ресурсов процессора \cite{inteltbb} (выбранная по причине наличия необходимости реализации параллельной версии алгоритма обратной трассировки лучей и совместимости с другими библиотеками потоков \cite{inteltbb}, что говорит об отсутствии конфликтов с потоками Qt);
	\item библиотека stl\_reader, предназначенная для обработки файлов, хранящих данные в формате STL, и преобразования их содержимого в пользовательские контейнеры \cite{stlreader} (выбранная по причине необходимости чтения STL-файлов заранее подготовленных моделей геометрических тел).
\end{itemize}

Для ускорения сборки программного обеспечения использовалась утилита CMake \cite{cmake}.

\section{Структура программного обеспечения}

В разрабатываемом программном обеспечении реализуются классы, описанные в диаграмме классов, которая представлена на рисунках \ref{img:uml_classes}, \ref{img:uml_classes_1},~\ref{img:uml_classes_2}.

\includeimage{uml_classes}{f}{h!}{0.9\textwidth}{Диаграмма классов (часть 1)}

\includeimage{uml_classes_1}{f}{h!}{0.9\textwidth}{Диаграмма классов (часть 2)}

\includeimage{uml_classes_2}{f}{h!}{0.9\textwidth}{Диаграмма классов (часть 3)}

Основные классы, реализуемые в программном обеспечении:

\begin{enumerate}
	\item {
		классы, служащие для управления сценой пользователем:
		\begin{itemize}
			\item Scene --- класс, описывающий набор объектов сцены;
			\item SceneManager --- класс инкапсулирования сцены;
			\item QtDrawer --- класс, предназначенный для изменения объекта QPixmap (экземпляр класса библиотеки Qt \cite{qt})
			\item Facade --- реализация паттерна <<фасад>>;
			\item NameCommand, SceneCommand --- реализация паттерна <<команда>>.
		\end{itemize}
}
	\item {
		классы, служащие для описания объектов сцены:
		\begin{itemize}
			\item Object --- базовый класс объекта сцены, определяющий его интерфейс;
			\item Model, Triangle --- базовые классы модели объекта сцены и ее декомпозиции;
			\item Camera --- наблюдатель (камера);
			\item Light --- точечный источник света;
		\end{itemize}
}
	\item {
		классы, обеспечивающие реализацию алгоритма обратной трассировки лучей: Ray, KDTree, BBox, Material, QVector.
}
\end{enumerate}

\clearpage

\section{Пользовательский интерфейс}