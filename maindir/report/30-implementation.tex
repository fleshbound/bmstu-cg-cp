\chapter{Технологическая часть}


% переделать и дополнить классами


Таким образом, возникает необходимость наличия следующих структур (совокупностей) данных для упрощения формального описания используемых алгоритмов:

\begin{itemize}
	\item Point --- информация о координатах в декартовой системе координат (значения $x$, $y$, $z$);
	\item Ray --- информация о луче: начало ($origin$), направление ($destination$);
	\item Material --- информация о спектральных характеристиках поверхности: интенсивность фонового излучения ($ambient$), диффузная составляющая интенсивности поверхности ($diffuse$), коэффициент отражения ($reflection$), коэффициент качества полировки ($polish$);
	\item Hitinfo --- информация о пересечении луча с поверхностью: значение параметра в соответствии с формулой \ref{eq:ray_t} ($t$), информация о материале поверхности ($material$), точка пересечения ($hitPoint$);
	\item BoundingBox --- информация об ограничивающем параллелепипеде: минимальные и максимальные координаты тела ($min\_p$, $max\_p$ соответственно);
	\item Object --- информация об объекте сцены: данные поверхности ($data$), центр ($center$), ограничивающий параллелепипед ($bbox$);
	\item kDNode --- информация об узле kD-дерева: дочерние узлы ($left$, $right$), признак листа ($isLeaf$), ограничивающий параллелепипед ($bbox$), соответствующая направлению декомпозиции ограничивающего параллелепипеда ось ($axis$), массив объектов ($objects$), количество объектов ($size$).
\end{itemize}

Камера (Camera) --- структура данных, содержащая информацию о наблюдателе: местоположение, направления взгляда и вертикальной оси.
Дополнительной информацией являются отношение сторон используемого экрана и область видимости (угол fov), необходимые для обеспечения возможности применения алгоритма обратной трассировки лучей.
Местоположение экрана зависит от камеры.