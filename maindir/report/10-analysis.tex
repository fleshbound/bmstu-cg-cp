\chapter{Аналитическая часть}

Модель, используемая в работе, требует соблюдения физических законов.

\section{Формализация задачи}

На рисунке \ref{img:task_diagram_a0} представлена диаграмма IDEF0 формализуемой задачи.

\includeimage{task_diagram_a0}{f}{h}{0.9\textwidth}{Диаграмма IDEF0}

Сцена состоит из набора сущностей:



\section{Физическая природа зеркальных поверхностей}

Луч (световой) --- узкий пучок света, представленный геометрической линией, вдоль которой распространяется свет \cite{тюрин2005физика}, \cite{оптика20036}. 

Если луч падает на поверхность в точку $A$ и отражается от нее, а через эту точку к поверхности проведена нормаль $n$, то углы, заключенные между нормалью и направлениями падающего, отраженного лучей ($a$, $b$), называются углами \textit{падения}, \textit{отражения} соответственно \cite{тюрин2005физика}, \cite{оптика20036}, \cite{rodionov}.
Описанные элементы представлены на рисунке \ref{img:reflection}.

\includeimage{reflection}{f}{h}{0.9\textwidth}{Отражение луча от поверхности}

Зеркало --- поверхность, от которой отражаются лучи.

Зеркальное отражение --- такое, для которого справедлива формула \ref{eq:reflect}:
\begin{equation}\label{eq:reflect}
	\alpha=\beta,
\end{equation}
где $\alpha$ --- угол падения, $\beta$ --- угол отражения, лучи и нормаль находятся в одной плоскости \cite{порев2002компьютерная}, \cite{оптика20036}, \cite{rodionov}.

Если от зеркала происходит зеркальное отражение, оно называется идеальным, иначе --- реальным \cite{порев2002компьютерная}. Связь интенсивности отраженного луча $I_b$ с интенсивностью падающего под углом $\alpha$ луча $I_a$ представлена в формулах \ref{eq:ideal_intense} и \ref{eq:intense_phong} для случаев идеального и реального зеркала соответственно \cite{порев2002компьютерная}, \cite{оптика20036}:

\begin{equation}\label{eq:ideal_intense}
	I_b=I_a
\end{equation}

\begin{equation}\label{eq:intense_phong}
	I_b=I_a K_b cos^p{\alpha},
\end{equation}
где $K_b$ --- некоторый числовой коэффициент пропорциональности, $p$ --- число от 1 до 200 \cite{порев2002компьютерная}.

\section{Существующие методы визуализации поверхностей}

Среди известных методов визуализации поверхностей существуют такие методы, как обратная трассировка лучей, 